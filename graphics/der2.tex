%%Created by jPicEdt 1.4.1_03: mixed JPIC-XML/LaTeX format
%%Mon Sep 14 09:20:08 CEST 2009
%%Begin JPIC-XML
%<?xml version="1.0" standalone="yes"?>
%<jpic x-min="-1.94" x-max="88.49" y-min="-2.63" y-max="47.9" auto-bounding="true">
%<multicurve arrow-head-length-scale= "1.5"
%	 arrow-head-width-minimum= "1.2"
%	 left-arrow= "head"
%	 fill-style= "none"
%	 points= "(5,47.9);(5,47.9);(4.99,-2.63);(4.99,-2.63)"
%	 />
%<multicurve arrow-head-length-scale= "1.6"
%	 arrow-head-width-minimum= "1.2"
%	 right-arrow= "head"
%	 fill-style= "none"
%	 points= "(-1.94,4.52);(-1.94,4.52);(88.49,4.52);(88.49,4.52)"
%	 />
%<text text-vert-align= "center-v"
%	 anchor-point= "(-0.65,37.9)"
%	 fill-style= "none"
%	 text-frame= "noframe"
%	 text-hor-align= "center-h"
%	 >
%$y$
%</text>
%<text text-vert-align= "center-v"
%	 anchor-point= "(70.76,-0.4)"
%	 fill-style= "none"
%	 text-frame= "noframe"
%	 text-hor-align= "center-h"
%	 >
%$x$
%</text>
%<multicurve fill-style= "none"
%	 points= "(18.83,27.73);(18.83,27.73);(38.74,38.44);(38.74,38.44)"
%	 />
%<multicurve fill-style= "none"
%	 points= "(27.49,32.19);(27.49,32.19);(27.49,30.86);(27.49,30.86)"
%	 />
%<multicurve fill-style= "none"
%	 points= "(24.89,30.86);(24.89,30.86);(27.49,30.86);(27.49,30.86)"
%	 />
%<text text-vert-align= "center-v"
%	 anchor-point= "(25.62,35.62)"
%	 fill-style= "none"
%	 text-frame= "noframe"
%	 text-hor-align= "center-h"
%	 >
%$P_2$
%</text>
%<text text-vert-align= "center-v"
%	 anchor-point= "(21.88,32.5)"
%	 fill-style= "none"
%	 text-frame= "noframe"
%	 text-hor-align= "center-h"
%	 >
%$P_1$
%</text>
%<pscurve fill-style= "none"
%	 closed= "false"
%	 curvature= "1;0.1;0"
%	 points= "(8.88,9.43);(9.31,13);(33.98,33.98);(60.8,17.02);(73.35,46.48);(75.08,57.64)"
%	 />
%<text text-vert-align= "center-v"
%	 anchor-point= "(58.06,29.52)"
%	 fill-style= "none"
%	 text-frame= "noframe"
%	 text-hor-align= "center-h"
%	 >
%$y = f(x)$
%</text>
%</jpic>
%%End JPIC-XML
%LaTeX-picture environment using emulated lines and arcs
%You can rescale the whole picture (to 80% for instance) by using the command \def\JPicScale{0.8}
\ifx\JPicScale\undefined\def\JPicScale{1}\fi
\unitlength \JPicScale mm
\begin{picture}(88.49,47.9)(0,0)
\linethickness{0.3mm}
\put(5,-2.63){\line(0,1){50.53}}
\put(5,47.9){\vector(0,1){0.12}}
\linethickness{0.3mm}
\put(-1.94,4.52){\line(1,0){90.43}}
\put(88.49,4.52){\vector(1,0){0.12}}
\put(-0.65,37.9){\makebox(0,0)[cc]{$y$}}

\put(70.76,-0.4){\makebox(0,0)[cc]{$x$}}

\linethickness{0.3mm}
\multiput(18.83,27.73)(0.22,0.12){89}{\line(1,0){0.22}}
\linethickness{0.3mm}
\put(27.49,30.86){\line(0,1){1.33}}
\linethickness{0.3mm}
\put(24.89,30.86){\line(1,0){2.6}}
\put(25.62,35.62){\makebox(0,0)[cc]{$P_2$}}

\put(21.88,32.5){\makebox(0,0)[cc]{$P_1$}}

\linethickness{0.3mm}
\qbezier(9.31,13)(13.28,20.65)(19.52,26.87)
\qbezier(19.52,26.87)(25.77,33.08)(33.98,33.98)
\qbezier(33.98,33.98)(41.73,33.12)(47.64,24.61)
\qbezier(47.64,24.61)(53.56,16.1)(60.8,17.02)
\qbezier(60.8,17.02)(67.93,19.93)(69.73,28.95)
\qbezier(69.73,28.95)(71.52,37.97)(73.35,46.48)
\put(58.06,29.52){\makebox(0,0)[cc]{$y = f(x)$}}

\end{picture}
