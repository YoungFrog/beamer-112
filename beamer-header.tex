\expandafter\includeonlylecture\expandafter{\currentcourse}
\mode<beamer>{\usetheme{Warsaw}} % put this one before \useoutertheme
\mode<presentation>{% = covers 4 modes : beamer, handout and two others.
  \useoutertheme{infolines}
  \setbeamertemplate{footline}{} %overwrite footline defined in infolines.
}
\setbeamertemplate{part page}{
  \begingroup
    \centering
    \vskip2em\par
    \begin{beamercolorbox}[sep=16pt,center]{part title}
      \usebeamerfont{part title}\insertpart\par
    \end{beamercolorbox}
  \endgroup
}

\AtBeginDocument{
  \setlength{\abovedisplayshortskip}{1pt plus 3pt}
  \setlength{\belowdisplayshortskip}{1pt plus 3pt}
  \setlength{\abovedisplayskip}{1pt plus 7pt}
  \setlength{\belowdisplayskip}{1pt plus 7pt}}
\usepackage[utf8]{inputenc} % Permet d'écrire avec les accents
\usepackage[T1]{fontenc}
\usepackage{babel}
\usepackage{filecontents}
\usepackage{relsize}
\usepackage{tabularx}
\frenchbsetup{IndentFirst=false} % must be removed if french is not used.
\usepackage{varwidth} % utilisé pour \Defn
\usepackage{amsthm}
\usepackage{xargs,xstring}
\usepackage{venndiagram}
\usepackage{amssymb}
\usepackage{amsmath}
\usepackage{amsfonts}
\usepackage{mathtools,booktabs}
%\usepackage{fourier}
\usepackage[light,frenchstyle]{kpfonts}
\usepackage{xfrac}
\usepackage{polynom}
\usepackage{tikz}
\usetikzlibrary{arrows}
\usetikzlibrary{decorations.text}
\usetikzlibrary{calc,through,fadings,decorations.pathreplacing}
\graphicspath{{graphics/}}
\makeatletter
\def\input@path{{graphics/}}
\makeatother
\usepackage{graphicx,color,epsfig,epstopdf}
%\usepackage{pstricks,pst-plot}
% \usepackage{chappg}
% \renewcommand{\chappgsep}{\thinspace--\thinspace}% looks slightly better to me (nr)
%\usepackage[language=auto]{biblatex}
\usepackage[inline]{asymptote}
\usepackage{siunitx,numprint}
\usepackage{hyperref}
%\usepackage{wasysym}
\usepackage{pifont}
\DeclareUrlCommand\email{\urlstyle{rm}}

%\addbibresource{MATHF112.bib}
%\newsubfloat{figure} % for subfigure

\makeindex
\renewcommand{\thesection}{\arabic{section}}
%\addtolength{\cftchapnumwidth}{1em} % Laissons la place aux chiffres Romains.

% \newtheoremstyle{defi}{7pt}{12pt}{}{}{\bfseries}{}{\newline}{}
% \theoremstyle{defi}
\theoremstyle{definition}
%\newtheorem{defn}{Définition}[chapter]
%\newtheorem{definition}[defn]{Définition}
\newtheorem*{exem}{Exemple}
\newtheorem*{question}{Question}
\newtheorem*{answer}{Réponse}
\newtheorem*{rappel}{Rappel}
\newtheorem*{enonce}{Énoncé}
%\newtheorem*{example}{Exemple}
\newtheorem*{remark}{Remarque}
\newtheorem*{remark*}{Remarque}
%%%%% Pour utiliser le package theorem %%%%%%%%%%%%%%%
% \newtheoremstyle{theom}{7pt}{12pt}{\it}{}{\bfseries}{}{\newline}{}
% \theoremstyle{theom}
\newtheorem*{exercise}{Exercice}%
\newtheorem*{exercise*}{Exercice}%

\theoremstyle{plain}
%\newtheorem{theorem}[defn]{Résultat}
\newtheorem{proposition}{Résultat}
\newtheorem{property}{Résultat}
%\newtheorem*{corollary}{Corollaire}
%\newtheorem{defi}[theorem]{Définition}
%%%%%%%%%%%%%%%%%%%%%%%%%%%%%%%%%%%%%%%%%%%%%%%%%%%%%%

%%%%%%%%%% Numérote correctement les équations %%%%%%%
\def\theequation{\thesection.\arabic{equation}}
%%%%%%%%%%%%%%%%%%%%%%%%%%%%%%%%%%%%%%%%%%%%%%%%%%%%%%
\newcommand{\dd}{\mathrm{d}}
\newcommand{\econstant}{\mathrm{e}}
\newcommand{\e}{\econstant}
\newcommand{\dom}{\mathop{\mathrm{dom}}}
\newcommand{\tg}{\mathop{\mathrm{tg}}}
\newcommand{\argch}{\mathop{\mathrm{argch}}}
\DeclareMathOperator{\ch}{ch}
\DeclareMathOperator{\sh}{sh}
\newcommand{\cotg}{\mathop{\mathrm{cotg}}}
\newcommand{\coseca}{\mathop{\mathrm{coseca}}}
\newcommand{\arctg}{\mathop{\mathrm{arctg}}}
\DeclareMathOperator{\arctanh}{arctanh}
\newcommand{\argsh}{\mathop{\mathrm{argsh}}}
\newcommand{\seca}{\mathop{\mathrm{seca}}}
\newcommand{\grad}{\mathop{\mathrm{grad}}}
\newcommand{\diver}{\mathop{\mathrm{div}}}
\newcommand{\rot}{\mathop{\mathrm{rot}}}
\newcommand{\biindice}[3]
{\renewcommand{\arraystretch}{0.5}
\begin{array}[t]{c}#1 \\
{\scriptstyle#2} \\
{\scriptstyle#3}
\end{array}
\renewcommand{\arraystretch}{1}}

\newcommand{\un}{\bf 1\!I}

\newcommand{\egs}[1]{\arraycolsep0.1pt \renewcommand{\arraystretch}{0.5}
\begin{array}[t]{c}=\\#1\end{array}\arraycolsep5pt \renewcommand{\arraystretch}{1}}

%%% MACROS DE SAMUEL FIORINI %%%

\newcommand{\N}{\mathbb{N}}
\newcommand{\Q}{\mathbb{Q}}
\newcommand{\R}{\mathbb{R}}
\newcommand{\Z}{\mathbb{Z}}

\newcommand\restr[2]{{\left.\kern-\nulldelimiterspace #1\vphantom{\big|}\right|_{#2}}}



\newcommandx*{\Defn}[3][1,3]{%
  %% 3 args:
  %% (optional) index entry (for \index)
  %% (mandatory) text used for typesetting (used in index if first arg not given)
  %% (optional) hypertarget name
%  \strictpagecheck % from memoir, checks that the marginpar is on the
                   % correct side. Requires one more LaTeX run at least.
  % \marginpar{\fbox{%
  %     \begin{varwidth}{\marginparwidth-2\fboxrule-2\fboxsep}
  %       \raggedright #2 %less hyphenation than \narrowragged
  %     \end{varwidth}
  %   }}
  % 
  \Defnemph{#2}%
  % \IfStrEq{#3}{}{\Defnemph{#2}}%{\hypertarget{#3}{\Defnemph{#2}}}%
  % \IfStrEq{#1}{}{% No first Arg
  %   \index{#2}%
  % }{%
  %   \index{#1}%
  % }%
}
\newcommand*{\ddx}[1][x]{\frac{\D}{\D #1}}
\newcommand*{\Defnemph}{\emph}
\newcommand*{\pardef}{\coloneqq} % Par d\'efinition.
\newcommand*{\defpar}{\eqqcolon} % Par d\'efinition.
\newcommand*{\eqpardef}{\pardef}
\newcommand*{\eqdefpar}{\defpar}
\newcommand*{\eqwant}{\stackrel{!}{=}}
\newcommand*{\eqmaybe}{\stackrel{?}{=}}
\newcommand{\set}[1]{\left\{#1\right\}} % Un ensemble { }
\newcommand*{\abs}[1]{\left\vert#1\right\vert} % Valeur absolue.
\newcommand*{\module}[1]{\left\vert#1\right\vert} % Valeur absolue.
\newcommand*{\norme}[1]{\left\Vert#1\right\Vert} % norme
\newcommand*{\ordre}[1]{\left\vert#1\right\vert} % L'ordre d'un \'el\'ement.
\newcommand{\conj}[2][]{\overline{\vphantom{#1}#2}}
\newcommand*{\scalprod}[2]{\left\langle #1,#2\right\rangle}
\newcommand*{\vecprod}{\times}
\newcommand*{\telque}{\mbox{~t.q.~}} % tel que, dans un ensemble.
\newcommand*{\ens}[1]{\mathbb{#1}} % Ensemble de nombres
\newcommand*{\var}[1]{\mathbf{#1}} % Vari\'et\'e
\newcommand*{\alg}[1]{\mathcal{#1}} % Alg\`ebre
\newcommand*{\RR}{\ens R}%
\newcommand*{\TT}{\ens T}% Tore ! Torus.
\newcommand{\sphere}{\var S}% Sph\`ere.
\newcommand{\CC}{\ens C}%
\newcommand{\ZZ}{\ens Z}%
\newcommand{\QQ}{\ens Q}%
\newcommand{\NN}{\ens N}%
\DeclareMathOperator{\im}{Im}
\DeclareMathOperator{\Id}{Id}
\DeclarePairedDelimiter{\intercc}{[}{]}
\DeclarePairedDelimiter{\interco}{[}{[}
\DeclarePairedDelimiter{\interoo}{]}{[}
\DeclarePairedDelimiter{\interoc}{]}{]}
\newcommand*{\limite}[3][]{\lim_{\substack{#2\rightarrow#3\\#1}}}
\newcommand*{\petito}[1]{\mathrm{o}(#1)}
\DeclarePairedDelimiter{\paren}{(}{)}
\DeclarePairedDelimiter{\braces}{\{}{\}}
\DeclarePairedDelimiter{\sqbracket}{[}{]}
\DeclareMathOperator{\pgcd}{pgcd}
\DeclareMathOperator{\ppcm}{ppcm}
\DeclareMathOperator{\Maj}{Maj}
\DeclareMathOperator{\Min}{Min}
\newenvironment{boxedalign}{%% FIXME: this is a quick hack not worth spreading
                            %% (e.g. does not obey [fleqn])
  \center
  \tabular{|l|}\hline\\
  $\aligned
}{%
  \endaligned$\\
  \\\hline
  \endtabular
  \endcenter
}
%% Points et vecteurs...
\newcommand{\fleche}[1]{\overrightarrow{#1}} % Flèche explicite.
\renewcommand{\vec}[1]{\mathbf{#1}} % adapt %% FIXME dans les slides de 14-15 j'ai mis une flèche à partir du cours 6.
\newcommand{\point}[1]{\mathbf{#1}} % adapt
%% FIXME: Utiliser \point{} et \vec{} de manière cohérente au travers
%% de tout le document. Je l'ai introduit en Géométrie anlaytique sans
%% annoncer la différence ()

% Dérivées partielles.
\newcommand*{\pder}[2]{\frac{\partial #1}{\partial #2}}
\DeclareMathOperator{\adh}{adh}
\DeclareMathOperator{\vol}{vol}

% Classe C^1, etc.
\newcommand{\Cclass}{\textup{C}}
\DeclareMathOperator{\Jac}{Jac}

%% FIXMELATER: L'idée était IIRC de mettre des informations "hors cours" mais potentiellement intéressantes pour certains étudiants. Peut on faire qqch d'utile avec ça ?
\newwrite\fyioutfile
\openout\fyioutfile=\jobname.fyi\relax
\newenvironment{fyi}[1][]{\color{gray}% #1 would be an explanation of what is being fyi'ed, in order to make a list of these somewhere. We could write to some aux file.
\IfStrEq{#1}{}{}{\write\fyioutfile{\string\item[Page \thepage] #1}}
}{}

%% Matrice vide pour que les étudiants complètent eux-même !
\newcommand{\blankmatrix}[2]{\begin{pmatrix}\hspace*{#1}\vspace{#2}\end{pmatrix}}

%% Faire en sorte de pouvoir récaptiuler les prérequis de chaque chapitre/section. Pour l'instant #1 n'est rien de particulier, mais on pourrait utiliser les labels pour être plus précis.
\newcommand{\requires}[1]{} %Suppress those.
\DeclarePairedDelimiter{\Span}{\langle}{\rangle}

%% Faire un système (d'égalités).
\newenvironment{systeme}{\left\{\begin{array}{r@{\,=\,}l}}{\end{array}\right.}
%\newenvironment{enonce}{\beginblock{Énoncé}}{\end{block}\pause}

\makeatletter
\renewcommand*\env@matrix[1][*\c@MaxMatrixCols c]{% this allows for a special optional argument to bmatrix (and others matrices environment)
  \hskip -\arraycolsep
  \let\@ifnextchar\new@ifnextchar
  \array{#1}}
\makeatother
\def\lnot{\neg}
\def\ldonc{\rightarrow}
\def\vers{\rightarrow}
\DeclareMathOperator{\interior}{int}
\newcommand{\mean}[1]{\bar{#1}}
\newcounter{numdecours}
\providecommand{\course}[1]{%
  \lecture{Cours #1}{Cours #1}
% \message{Frontière de cours en page \thepage}
}
\def\basep{\point o}
\def\noqed{\renewcommand{\qedsymbol}{}}
%\makeatletter{}
%\newcommand{\evaluateAt@}[1]{\left.#1\right.}
\newcommand{\evaluatedAt}[2]{{#1}_{\Big\vert #2}}
%\makeatother

\newcounter{currentstage}
\newcommand\dopolylongdiv[5][]{%
   \begingroup
       \setcounter{currentstage}{#4}
       \addtocounter{currentstage}{-1}
       \loop \ifnum \value{currentstage}<#5\relax
          \addtocounter{currentstage}{1}%
          \only<\value{currentstage}>{\polylongdiv[stage=\value{currentstage},#1]{#2}{#3}}
       \repeat
   \endgroup
}
\DeclareMathOperator{\Mat}{Mat}
\DeclareMathOperator{\Tr}{Tr}
\DeclareMathOperator{\cis}{cis}
\newcommand{\tikzmark}[2]{{\shorthandoff{:;!?}\tikz[remember picture,baseline] \node [anchor=base] (#1) {$#2$};}}

\newcommand{\DrawLine}[3][]{%
  \begin{tikzpicture}[overlay,remember picture]
    \draw[#1] (#2.north) -- (#3.south);
  \end{tikzpicture}
}
\newcommand{\DrawHLine}[3][]{%
  \begin{tikzpicture}[overlay,remember picture]
    \draw[#1] (#2.west) -- (#3.east);
  \end{tikzpicture}
}
\newcommand{\DrawBox}[3][]{%
  \begin{tikzpicture}[overlay,remember picture]
    \draw[#1] (#2.north west) rectangle (#3.south east);
  \end{tikzpicture}
}
\newcommand*{\transpose}[1]{{\vphantom{#1}}^{\mathit t}{\!\/#1}}
\expandafter\def\csname Parent2\endcsname{} %% keep beamer + hyperref happy when processing a region that begins with a subsection without a section.
\newenvironment{parenthèse}{\mode<handout:0>{\setbeamercolor{normal text}{bg=black!10}}}{}
\renewcommand{\vec}[1]{\fleche{#1}}
\newcommand{\somme}[1]{\sum_{#1 = 1}^n}
\definecolor{ulb-ulb}{cmyk}{     1    , 0.75 , 0.2  , 0.18}%287C
\definecolor{ulb-epb}{cmyk}{     0    , 0    , 0    , 1}%Black
\definecolor{ulb-droit}{cmyk}{   0    , 1    , 0    , 0}%Magenta
\definecolor{ulb-archi}{cmyk}{   0.52 , 0    , 0.86 , 0}%368C
\definecolor{ulb-fsp}{cmyk}{     1    , 0    , 0.11 , 0.2}%313C
\definecolor{ulb-philo}{cmyk}{   0.23 , 0.16 , 0.13 , 0.46}%Cool gray 8C
\definecolor{ulb-med}{cmyk}{     0    , 0.95 , 1    , 0}%485C
\definecolor{ulb-psycho}{cmyk}{  0.99 , 0.50 , 0    , 0}%300C
\definecolor{ulb-sciences}{cmyk}{0.68 , 0.76 , 0    , 0}%272C
\definecolor{ulb-sbs}{cmyk}{     0    , 0.40 , 0.90 , 0}%1375C
\definecolor{ulb-iee}{cmyk}{     0.10 , 0.69 , 0    , 0.4}%293C
\definecolor{ulb-pharma}{cmyk}{  0.92 , 0    , 0.84 , 0.20}%348C
